\section{PRATA GORUTUBA}
A "Prata Gorutuba" é um tipo de banana selecionado a partir de uma mutação espontânea da variedade Prata Anã, que é cultivada no Norte de Minas Gerais. De acordo com Lopez \& Espinosa (1995), as bananeiras têm alta eficiência em produzir uma grande quantidade de biomassa em um curto período de tempo, o que demanda altas taxas de nutrientes. Portanto, garantir a nutrição adequada das bananeiras é de extrema importância para a cadeia produtiva.

A grande quantidade de fertilizantes necessária se deve não apenas à alta demanda de nutrientes absorvidos e exportados pelos frutos, mas também ao fato de que os solos da maioria das regiões produtoras geralmente apresentam baixa fertilidade (Borges \& Oliveira, 2000). A bananeira é uma planta sensível a desequilíbrios nutricionais. Portanto, é essencial manter o equilíbrio dos nutrientes no solo, evitando assim o consumo excessivo de um elemento, o que pode levar à deficiência de outros (Gutierrez, 1983). A demanda por nutrientes pela planta depende da taxa de crescimento e da eficiência em converter os nutrientes absorvidos em biomassa.