
\chapter{REFERENCIAL TEÓRICO}
\section{A bananeira}
A bananeira é uma planta de crescimento rápido que necessita de quantidades adequadas de nutrientes disponíveis no solo para seu desenvolvimento e produção. Embora parte das necessidades nutricionais possa ser suprida pelo solo e pelos resíduos das colheitas, frequentemente é necessário aplicar calcário e fertilizantes para garantir uma produção rentável economicamente \cite{culturabanana}.

A quantidade de nutrientes que precisa ser aplicada à variedade de bananeira depende de fatores como o potencial produtivo da planta, a densidade populacional, o estado fitossanitário e, principalmente, o balanço de nutrientes no solo e o sistema radicular, que influenciam a absorção dos nutrientes. A demanda por nutrientes é elevada devido à grande quantidade exportada durante a colheita dos cachos de banana \cite{culturabanana}.

O potássio (K) e o nitrogênio (N) são os nutrientes mais absorvidos e necessários para o crescimento e produção da bananeira, seguidos pelo magnésio (Mg) e pelo cálcio (Ca). Em menor grau de absorção, encontram-se os nutrientes enxofre (S) e fósforo (P) \cite{culturabanana}.

Entre os micronutrientes estudados, o boro (B) e o zinco (Zn) são os mais absorvidos, principalmente pela variedade de bananeira "Terra", seguidos pelo cobre (Cu) \cite{culturabanana}.

A bananeira é uma cultura que demanda uma quantidade significativa de nutrientes durante a adubação, porém também contribui com uma boa quantidade de nutrientes para o solo. É importante ressaltar que a necessidade de nutrientes varia entre as diferentes variedades de bananeira \cite{culturabanana}.

\section{Grande Naine}
A Grande Naine é uma variedade mutante da banana Nanica, originária da ilha de Martinica. Embora seu pseudocaule seja semelhante ao da Nanica, possui manchas escuras e tamanho médio, com altura variando entre 2,5 e 3 metros. A roseta foliar é um pouco mais solta, o que diminui o problema de engasgamento da inflorescência. As folhas são um pouco maiores do que as da Nanica, apresentando cores menos intensas e menor cerosidade. O cacho é mais peludo e um pouco mais longo e tem uma forma ligeiramente cônica, com frutos delgados, longos e curvados, tendo a extremidade arredondada, pedicelos curtos e polpa madura com um sabor muito doce. As primeiras fileiras de bananas no cacho têm almofadas curtas e as bananas são menos curvas em comparação com a banana Nanica, o que facilita seu acondicionamento em caixas de embalagem. A ráquis é reta, sendo comum que o primeiro terço esteja quase livre de resíduos florais masculinos, enquanto o restante está preenchido com esses resíduos parcialmente cobertos por brácteas. O coração da Grande Naine é maior do que o da Nanica. O peso fresco do cacho varia de 15kg a 30kg, dependendo dos níveis de nutrição aplicados, uma vez que essa cultivar responde bem à adubação \cite{cultivodebananeira}.

É uma das variedades cultivadas para a comercialização no mercado internacional. Seu porte menor em comparação com a bananeira Nanicão ajuda a reduzir os danos causados por ventos fortes. No entanto, as perdas devido às últimas fileiras de bananas não estarem dentro dos padrões são significativas. Para compensar essa deficiência, às vezes é necessário remover até as três últimas fileiras. A cultivar possui alta capacidade de resposta em condições de alta tecnologia, mas não tem a mesma resistência que a cultivar Nanicão \cite{cultivodebananeira}.

Essa variedade apresenta suscetibilidade à sigatoka amarela \cite {rocha2008epidemiologia, cultivodebananeira} e também a sigatoka negra e também aos nematoides e ao moleque-da-bananeira, entretanto é resistente ao fungo da fusariose \cite{cultivodebananeira}

\section{PRATA GORUTUBA}
A "Prata Gorutuba" é um tipo de banana selecionado a partir de uma mutação espontânea da variedade Prata Anã, que é cultivada no Norte de Minas Gerais \cite{pratagorutuba1}. De acordo com Lopez \& Espinosa (1995), as bananeiras têm alta eficiência em produzir uma grande quantidade de biomassa em um curto período de tempo, o que demanda altas taxas de nutrientes. Portanto, garantir a nutrição adequada das bananeiras é de extrema importância para a cadeia produtiva.

A grande quantidade de fertilizantes necessária se deve não apenas à alta demanda de nutrientes absorvidos e exportados pelos frutos, mas também ao fato de que os solos da maioria das regiões produtoras geralmente apresentam baixa fertilidade (Borges \& Oliveira, 2000). A bananeira é uma planta sensível a desequilíbrios nutricionais. Portanto, é essencial manter o equilíbrio dos nutrientes no solo, evitando assim o consumo excessivo de um elemento, o que pode levar à deficiência de outros (Gutierrez, 1983). A demanda por nutrientes pela planta depende da taxa de crescimento e da eficiência em converter os nutrientes absorvidos em biomassa.

\section{Maçã}
Essa cultivar possui frutos de excelente qualidade e é altamente aceita no mercado consumidor devido ao seu aroma e sabor semelhantes à maçã. A planta tem um porte que varia de 3,5 a 4 metros de altura e um diâmetro de 30 a 35 centímetros na base do pseudocaule. Na parte mais alta do pseudocaule, há algumas manchas esporádicas e irregulares, que são quase pretas. As folhas são levemente opacas e têm uma disposição inclinada para baixo, dando à planta a aparência de um guarda-chuva aberto. A maior curvatura das folhas ocorre nos primeiros metros a partir do pecíolo. A inflorescência é pequena e o cacho tem tamanho médio e é relativamente fino \cite{cultivodebananeira}.

O peso fresco do cacho varia de 10kg a 12kg e é composto por seis a oito pencas. As pencas estão bem espaçadas ao longo da ráquis, sendo que as primeiras têm em média 18 bananas, enquanto as últimas têm de seis a oito bananas. O comprimento dos frutos varia de 10cm a 18cm. A casca é fina, exalando um suave aroma, e quando maduras, as bananas apresentam uma coloração amarela intensa. A polpa é levemente adocicada, extremamente macia, quase com uma textura farinácea, e tem uma coloração branca. Ocasionalmente, podem ser encontradas sementes férteis no interior dos frutos. A ráquis é bastante longa, de espessura mediana e está livre de resíduos florais \cite{cultivodebananeira}.

Um dos principais desafios para o cultivo dessa cultivar é sua alta suscetibilidade à fusariose-da-bananeira, exigindo uma adubação adequada com Zn, Ca, Mg e P. Essa cultivar tem sido utilizada como cultura de desbravamento em regiões do interior do Brasil devido ao seu período de produção muito curto, limitado a apenas uma a três colheitas, devido à presença dessa doença fúngica. Recomenda-se realizar o plantio em solos que não tenham sido utilizados para o cultivo de bananeiras nos últimos trinta anos, e preferencialmente utilizando mudas produzidas por meio de biotecnologia. Além disso, essa cultivar é suscetível ao ataque do moleque-da-bananeira e nematoides, mas apresenta resistência à sigatoka amarela \cite{cultivodebananeira}.

\section{BRS Princesa}
É um híbrido desenvolvido pela Embrapa - Mandioca e Fruticultura. Trata-se do cruzamento entre as variedades Yangambi n.° 2 (AAB) e M53 (AA). Os frutos desse híbrido se assemelham aos da cultivar Maçã \cite{cultivodebananeira}. 

As plantas têm um porte alto, variando de 3 a 5,5 metros, com um pseudocaule vigoroso que mede entre 20 a 35 centímetros de diâmetro. A massa fresca do cacho varia de 10 a 16 quilogramas, com frutos que têm um comprimento de 10 a 15 centímetros e um diâmetro de 30 a 35 milímetros. O sabor desses frutos é semelhante ao da banana Maçã. Um detalhe importante nesse híbrido é as resistências ao mal-do-panamá e as sigatokas (amarela e negra) \cite{cultivodebananeira}.


\section{Bokashi}
O bokashi é um termo japonês que significa "matéria fermentada" ou "matéria orgânica fermentada". É um método tradicional de compostagem utilizado na agricultura orgânica e na jardinagem para melhorar a fertilidade do solo e fornecer nutrientes para as plantas.

O processo de produção do bokashi envolve a fermentação de resíduos orgânicos, como restos de alimentos, aparas de grama, folhas, cascas de frutas e vegetais, entre outros materiais ricos em nutrientes. Esses resíduos são misturados com um inoculante, que geralmente contém microorganismos eficientes (como bactérias e fungos benéficos) e materiais que estimulam a decomposição.

Durante a fermentação, os microorganismos presentes no bokashi quebram a matéria orgânica, convertendo-a em nutrientes solúveis que as plantas podem absorver mais facilmente. Além disso, o bokashi ajuda a melhorar a estrutura do solo, aumentar a capacidade de retenção de água e promover a atividade microbiana benéfica no solo.

O bokashi pode ser usado como um adubo orgânico, sendo adicionado diretamente ao solo ou utilizado como cobertura do solo. Também pode ser usado na compostagem tradicional, misturado com outros materiais orgânicos para acelerar o processo de decomposição.

O uso regular de bokashi pode melhorar a saúde do solo, aumentar a produtividade das plantas, reduzir a necessidade de fertilizantes químicos e contribuir para a sustentabilidade agrícola. É uma técnica popular entre os praticantes da agricultura orgânica e da permacultura.
