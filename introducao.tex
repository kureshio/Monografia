\chapter{INTRODUÇÃO}

A banana (Musa spp.) é uma das frutas mais consumidas no mundo, sendo produzida na maioria dos países tropicais. Em 2021, a produção mundial atingiu aproximadamente 125 milhões de toneladas, com a Índia como o principal país produtor. Em segundo lugar, com 9,4\% da produção total, vem a china, o Brasil ocupa a quarta maior produção com 5,5\% da produção total \cite{banana}. A banana é uma fruta que tem o sabor mediamente doce e textura firme \cite{MATSUURA2004}. Acredita-se que a banana seja nativa do sudeste da ásia \cite{1956TaOo}.

%A banana é uma das frutas mais consumidas no Brasil, com uma produção nacional próxima de 7 milhões de toneladas no ano de 2021, isso torna o Brasil o quarto maior produtor de banana do mundo \cite{banana} Ela também é a fruta fresca mais consumida no mundo. No Brasil o setor gera mais de 500.000 empregos diretos, em Minas Gerais são 60.000 postos de trabalho

Segundo o IBGE, em 2020 foram produzidas 6,7 milhões de toneladas no país, em uma área de mais de 456 mil hectares, torna o Brasil o quarto maior produtor de banana do mundo \cite{banana}. No entanto, a cultura da banana enfrenta diversos desafios, como pragas, doenças, deficiências nutricionais e baixa produtividade. Uma forma de melhorar a qualidade e a quantidade da produção é o uso de adubos orgânicos, que fornecem nutrientes essenciais para as plantas, além de melhorar as propriedades físicas, químicas e biológicas do solo.

Um dos adubos orgânicos que vem ganhando destaque é o bokashi, que significa "matéria orgânica fermentada" em japonês. O bokashi é um composto obtido pela mistura de diversos materiais orgânicos, como farelos, esterco, cinzas, terra e micro-organismos eficazes (EM), que são responsáveis pela fermentação anaeróbica do material. O bokashi apresenta diversas vantagens, como a rápida decomposição, a liberação controlada de nutrientes, a inoculação de micro-organismos benéficos no solo e a redução de odores e patógenos.

O objetivo deste trabalho foi avaliar o efeito de diferentes tipos de adubos orgânicos, incluindo o bokashi, no crescimento e no desenvolvimento de mudas de banana. Foram utilizados sete tratamentos: testemunha (sem adubo), bokashi, esterco bovino, esterco de galinha, composto orgânico, torta de mamona e húmus de minhoca. As mudas foram cultivadas em vasos por 90 dias e avaliadas quanto à altura, diâmetro do colo, número de folhas, área foliar, massa seca da parte aérea e da raiz e índice de qualidade de Dickson. Os resultados foram submetidos à análise de variância e ao teste de Tukey a 5% de probabilidade.

%²: Blog | Broto: Seu Jeito Digital de Fazer Agro! (2021). Produção de banana no Brasil. https://blog.broto.com.br/producao-de-banana-no-brasil/

%⁴: Caracterização dos principais polos de produção de banana no Brasil ... (2015). Embrapa Mandioca e Fruticultura - Documentos. https://www.embrapa.br/busca-de-publicacoes/-/publicacao/967182/caracterizacao-dos-principais-polos-de-producao-de-banana-no-brasil

%Origem: conversação com o Bing, 06/07/2023
%(1) TABELA - Produção brasileira de banana em 2021 Área Colhida Produção .... http://www.cnpmf.embrapa.br/Base_de_Dados/index_pdf/dados/brasil/banana/b1_banana.pdf.
%(2) Caracterização dos principais polos de produção de banana no Brasil .... https://www.embrapa.br/busca-de-publicacoes/-/publicacao/967182/caracterizacao-dos-principais-polos-de-producao-de-banana-no-brasil.
%(3) Blog | Broto: Seu Jeito Digital de Fazer Agro!. https://blog.broto.com.br/producao-de-banana-no-brasil/.
%(4) Banana - Portal Embrapa. https://www.embrapa.br/mandioca-e-fruticultura/cultivos/banana.