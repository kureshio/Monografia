\chapter{INTRODUÇÃO}
A banana, fruto de sabor mediamente doce e textura firme \cite{MATSUURA2004}. Tem uma história fascinante e uma origem que remonta a milhares de anos. Acredita-se que a banana seja nativa do sudeste da Ásia, especialmente das regiões que atualmente compreendem a Malásia, Indonésia e Filipinas \cite{1956TaOo}. Ao longo dos séculos, ela se espalhou por diferentes partes do mundo. As primeiras referências à banana remontam a textos antigos, como registros egípcios e indianos. A popularidade dessa fruta cresceu gradualmente, e a expansão das rotas comerciais no século XIX contribuiu para a disseminação da banana em diversas regiões do globo. Hoje em dia, a banana é cultivada em mais de 150 países, tornando-se uma das frutas mais consumidas e comercializadas em todo o mundo.

A banana é uma das frutas mais consumidas no Brasil, com uma produção nacional próxima de 7 milhões de toneladas no ano de 2021, isso torna o Brasil o quarto maior produtor de banana do mundo \cite{banana} Ela também é a fruta fresca mais consumida no mundo. No Brasil o setor gera mais de 500.000 empregos diretos, em Minas Gerais são 60.000 postos de trabalho