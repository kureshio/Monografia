\chapter{Resultado e Discussão}

Os resultados do experimento com as quatro variedades de banana (Naine, Gorutuba, Maçã e Princesa) e os sete diferentes tratamentos são apresentados a seguir. As características avaliadas foram o diâmetro do pseudocaule, a altura da planta e a quantidade de folhas. Os dados foram analisados utilizando análise de variância (ANOVA) e as diferenças entre os tratamentos foram consideradas significativas quando p < 0,05.
\section{Grande Naine}

Para a variedade Grande Naine, a análise de variância não revelou diferenças significativas nas características estudadas entre os tratamentos.

\begin{table}[!htb]
 	\begin{center}
 		\caption{Resultado da análise de variância das características testadas para a variedade Grande-naine}.
	 	\begin{tabular*}{\textwidth}{@{\extracolsep{\fill}}lcccr}
 		\toprule
 		\toprule
 		\textbf{Caracteres} & \textbf{F}  & \textbf{QM(Erro)} & \textbf{Média Geral} &\textbf {CV(\%)} \\
		\hline
		Diâmetro do pseudocaule &0.09	&2.89		&9.03 	&22.78 \\
		Altura de planta 		&0.38 	&5.17 		&8.60 	&26.46 \\
		Quantidade de folhas 	&0.17  	&1.29 		&5.40 	&20.97\\
		\hline
		\hline
 		\end{tabular*}
 		\centering {\small Fonte: fonte da tabela} %Fonte da tabela
 	\end{center}
\end{table}

\section{Gorutuba}
Análise de variância Não revelou diferenças significativas na quantidade de folhas entre os tratamentos
Altura de planta não mostrou diferenças entre alguns tratamentos
diÂmetro, não mostrou diferenças significativas

\begin{table}[!htb]
	\begin{center}
		\caption{Resultado da análise de variância das características testadas para a variedade Gorutuba}.
		\begin{tabular*}{\textwidth}{@{\extracolsep{\fill}}lcccr}
			\toprule
			\toprule
			\textbf{Caracteres} & \textbf{F}  & \textbf{QM(Erro)} & \textbf{Média Geral} &\textbf {CV(\%)} \\
			\hline
			Diâmetro do pseudocaule & 0.79 & 3.47 & 8.76 & 21.27\\
			Altura de planta 		& 5.96 & 0.0003 & 11.36 & 24.69\\
			Quantidade de folhas 	& 0.69 & 0.99 & 4.5 & 21.70\\
			\hline
			\hline
		\end{tabular*}
		\centering {\small Fonte: fonte da tabela} %Fonte da tabela
	\end{center}
\end{table}

\section{Princesa}
\begin{table}[!htb]
	\begin{center}
		\caption{Resultado da análise de variância das características testadas para a variedade Princesa}.
		\begin{tabular*}{\textwidth}{@{\extracolsep{\fill}}lcccr}
			\toprule
			\toprule
			\textbf{Caracteres} & \textbf{F}  & \textbf{QM(Erro)} & \textbf{Média Geral} &\textbf {CV(\%)} \\
			\hline
			Diâmetro do pseudocaule & 1.99 & 2.89 & 10.66 & 16.78\\ 
			Altura de planta 		& 1.67 & 12.67 & 15.64 & 12.55\\
			Quantidade de folhas 	& 0.54 & 0.82 & 4.3 & 20.57\\
			\hline
			\hline
		\end{tabular*}
		\centering {\small Fonte: fonte da tabela} %Fonte da tabela
	\end{center}
\end{table}

\section{Maçã}
\begin{table}[!htb]
	\begin{center}
		\caption{Resultado da análise de variância das características testadas para a variedade Princesa}.
		\begin{tabular*}{\textwidth}{@{\extracolsep{\fill}}lcccr}
			\toprule
			\toprule
			\textbf{Caracteres} & \textbf{F}  & \textbf{QM(Erro)} & \textbf{Média Geral} &\textbf {CV(\%)} \\
			\hline
			Diâmetro do pseudocaule &&&& 16.78\\ 
			Altura de planta 		&&& 15.64 & 12.55\\
			Quantidade de folhas 	&&& 4.3 & 20.57\\
			\hline
			\hline
		\end{tabular*}
		\centering {\small Fonte: fonte da tabela} %Fonte da tabela
	\end{center}
\end{table}
