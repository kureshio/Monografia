\chapter{Materiais e Métodos}

O trabalho foi conduzido de agosto de 2023 à dezembro de 2023, no setor de fruticultura, pertencente ao departamento de agronomia da Universidade Federal de Lavras --- UFLA, que está localizada no municípo de Lavras, que fica a 855 \unit{\metre} de altitude e 21º15'00' de latitude sul. O clima da região é subtropical.

\section{Material Genético}
Foram utilizadas quatro cultivares,adquiridas da Empresa Multiplanta,om cerca de 7 \unit{\centi\metre} de altura. Essas mudas foram transplantadas  para tubetes contendo substrato comercial da marca Carolina Soil, sendo 70 mudas da variedade Grande Naine, 70 da Prata-Gorutuba, 21 da Maçã e 21 da Princesa. Todas as mudas receberam os mesmos tratamentos e condições ambientais durante a condução do experimento dentro de uma casa de vegetação climatizada com uma chuva ao longo do dia.

\section{Organização do experimento}
O experimento foi realizado utilizando o delineamento inteiramente casualisado (DIC), com sete tratamentos cada cultivar. Sendo que a variedade Grande Naine teve 10 plantas para cada tratamento, Prata-Gorutuba também teve 10 plantas para cada tratamento, Maçã e Princesa tiveram três plantas para cada tratamento.

\section{Lista de tratamentos}

\begin{itemize}
	\item T1: 85 \unit{\gram} de substrato e 5 \unit{\gram} do adubo Bokashi;
	\item T2: 85 \unit{\gram} de substrato, 5 \unit{\gram} do adubo Bokashi e 5 \unit{\gram} do condicionador de solo lithothamnium;
	\item T3: 85 \unit{\gram} de substrato, e 5 \unit{\gram} do adubo Bokashi e 5 \unit{\gram} do adubo Baks da Empresa Verde;
	\item T4: 57 \unit{\gram} de substrato e 28 \unit{\gram} do adubo natural de gado;
	\item T5: 57 \unit{\gram} de substrato, 28 \unit{\gram} do adubo natural de gado e 5 \unit{\gram} do condicionador de solo lithothamnium;
	\item T6: 57 \unit{\gram} de substrato, 28 \unit{\gram} do adubo natural de gado e 5 \unit{\gram} do adubo Baks da Empresa Verde;
	\item T7: 85 \unit{\gram} de substrato e 5 \unit{\gram} de adubo químico.
\end{itemize}

\section{Características Avaliadas}
As seguintes características foram avaliadas: Altura de planta, diâmetro do pseudocaule, número de folhas, largura e comprimento de folha.
