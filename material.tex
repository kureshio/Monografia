\chapter{Materiais e Métodos}

O trabalho foi conduzido de agosto de 2023 à dezembro de 2023, no setor de fruticultura, pertencente ao departamento de agronomia da Universidade Federal de Lavras --- UFLA, que está localizada no municípo de Lavras, que fica a 855 \unit{\metre} de altitude e 21º15'00' de latitude sul. O clima da região é subtropical.

\section{Material Genético}
Foram utilizadas quatro cultivares,adquiridas da Empresa Multiplanta,om cerca de 7 \unit{\centi\metre} de altura. Essas mudas foram transplantadas  para tubetes contendo substrato comercial da marca Carolina Soil, sendo 70 mudas da variedade Grande Naine, 70 da Prata-Gorutuba, 21 da Maçã e 21 da Princesa. Todas as mudas receberam os mesmos tratamentos e condições ambientais durante a condução do experimento dentro de uma casa de vegetação climatizada com uma chuva ao longo do dia.

\section{Organização do experimento}
O experimento foi realizado utilizando o delineamento inteiramente casualisado (DIC), com sete tratamentos cada cultivar. Sendo que a variedade Grande Naine teve 10 plantas para cada tratamento, Prata-Gorutuba também teve 10 plantas para cada tratamento, Maçã e Princesa tiveram três plantas para cada tratamento.

\section{Lista de tratamentos}

\begin{itemize}
	\item T1: 85 \unit{\gram} de substrato e 5 \unit{\gram} do adubo Bokashi;
	\item T2: 85 \unit{\gram} de substrato, 5 \unit{\gram} do adubo Bokashi e 5 \unit{\gram} do condicionador de solo lithothamnium;
	\item T3: 85 \unit{\gram} de substrato, e 5 \unit{\gram} do adubo Bokashi e 5 \unit{\gram} do adubo Baks da Empresa Verde;
	\item T4: 57 \unit{\gram} de substrato e 28 \unit{\gram} do adubo natural de gado;
	\item T5: 57 \unit{\gram} de substrato, 28 \unit{\gram} do adubo natural de gado e 5 \unit{\gram} do condicionador de solo lithothamnium;
	\item T6: 57 \unit{\gram} de substrato, 28 \unit{\gram} do adubo natural de gado e 5 \unit{\gram} do adubo Baks da Empresa Verde;
	\item T7: 85 \unit{\gram} de substrato e 5 \unit{\gram} de adubo químico.
\end{itemize}

\section{Características Avaliadas}
As seguintes características foram avaliadas: Altura de planta, diâmetro do pseudocaule, número de folhas, largura e comprimento de folha.

\section{Estatística}
O software R foi utilizado para realizar a análise estatística dos dados coletados. A avaliação dos dados foi realizada utilizando três técnicas principais: análise de variância, teste de Shapiro-Wilk e teste de Levene. Essas análises desempenharam um papel fundamental na obtenção de resultados confiáveis e na interpretação dos efeitos e diferenças observadas.

A análise de variância (ANOVA) foi aplicada para examinar a existência de diferenças significativas entre as médias de diferentes tratamentos de cada variedade de banana. Esse teste estatístico permitiu-me determinar se as diferenças observadas entre os grupos eram estatisticamente significativas, ajudando a estabelecer relações causais e identificar os fatores que contribuíram para as variações observadas.

Além disso, utilizei o teste de Shapiro-Wilk para avaliar a normalidade da distribuição dos dados em cada grupo. Esse teste é amplamente utilizado para verificar se uma amostra segue uma distribuição normal. Ao realizar o teste de Shapiro-Wilk, pude verificar se os pressupostos estatísticos subjacentes à análise de variância foram atendidos, fornecendo uma validação importante para a aplicação do teste.

Adicionalmente, empreguei o teste de Levene para avaliar a homogeneidade das variâncias entre os grupos. Esse teste permite verificar se as variâncias são iguais entre os grupos e é importante para garantir que a análise de variância seja aplicada corretamente. Ao identificar diferenças significativas nas variâncias entre os grupos, pude tomar medidas adequadas para corrigir possíveis violações dos pressupostos estatísticos e garantir resultados mais precisos.

Para lidar com as características dos dados que não apresentavam distribuição normal, foi aplicada a transformação Box-Cox. Essa transformação é uma técnica estatística que é aplicada às variáveis não normais com o objetivo de ajustar sua distribuição para uma forma mais próxima da normalidade. Essa transformação é baseada em uma função que depende de um parâmetro lambda (\lambda).