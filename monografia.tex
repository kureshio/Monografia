\documentclass{uflamon}          % classe base para a monografia

%==============================================================================
% Utilizacao de pacotes
\usepackage[T1]{fontenc}         % usa fontes postscript com acentos
\usepackage[brazil]{babel}       % hifenização e títulos em português do Brasil
\usepackage[utf8]{inputenc}     % permite edição direta com acentos
\usepackage{amsmath}             % pacote da AMS para Matemática Avançada
\usepackage{amssymb}             % símbolos extras da AMS
\usepackage{latexsym}            % símbolos extras do LaTeX
\usepackage{graphicx}            % para inserção de gráficos
\usepackage{listings}            % para inserção de código
\usepackage{fancyvrb}            % para inserção de saídas de comandos
%\usepackage{enumerate}           % para personalizar lista enumeradas 
											%(incluso na classe)
\usepackage{longtable}           % para tambelas muito grandes NOVO!!!!

\usepackage{colortbl} % cores em tabelas
\newcolumntype{Z}{|>{\columncolor[gray]{0.9}}l|} %cor cinza em células
%\usepackage{array} % já incluso na classe
\newcolumntype{L}[1]{>{\raggedright\let\newline\\\arraybackslash\hspace{0pt}}m{#1}}
\newcolumntype{C}[1]{>{\centering\let\newline\\\arraybackslash\hspace{0pt}}m{#1}}
\newcolumntype{R}[1]{>{\raggedleft\let\newline\\\arraybackslash\hspace{0pt}}m{#1}}
\usepackage{multirow} % para juntar duas linhas em uma só

\usepackage{multicol} % para uso de várias colunas

% cores para os links cruzados
\usepackage{color}
\definecolor{rltred}{rgb}{0.2,0,0}
\definecolor{rltgreen}{rgb}{0,0.2,0}
\definecolor{rltblue}{rgb}{0,0,0.2}

\usepackage[colorlinks=true,
            urlcolor=rltblue,       % \href{...}{...} external (URL)
            filecolor=rltgreen,     % \href{...} local file
            linkcolor=rltred,       % \ref{...} and \pageref{...}
            citecolor=rltgreen,
            pdftitle={Exemplo de Uso da Classe Uflamon},
          pdfauthor={Joaquim Quinteiro Uchôa},
          pdfsubject={Este texto tem por objetivo servir de exemplo da classe Uflamon.},
          pdfkeywords={Comunicação Científica. 2. Pesquisa . 3. Pesquisa Científica. 
 					 4. Redação. 5. Monografia.}%
]{hyperref} % para referência cruzadas
%\usepackage{hyperref}            % para referência cruzadas
\usepackage{subfigure}           % figuras dentro de figuras
\usepackage{caption}            % remodelando o formato dos títulos de 
                                 % tabelas e figuras

% configuração padrão do listings   
\lstset{
   language=Java,
   extendedchars=true,
   tabsize=3,
   basicstyle=\footnotesize\ttfamily,
   stringstyle=\em,
   showstringspaces=false 
}

% para referências de acordo com a ABNT
% precisa instalar o abntex2 antes!!!
% http://abntex.codigolivre.org.br/
% comente se pretende usar outro padrão

%abnt-emphasize=bf coloca o título das bibliografias em negrito
%abnt-thesis-year=both
\usepackage[alf,abnt-etal-cite=3,abnt-etal-list=3,abnt-url-package=url,abnt-emphasize=bf]{abntex2cite}

% evite usar o hyperref com abntex, pode dar caca em urls... no linha anterior, informo
% para incluir urls usando o pacote url e não o hyperref
%
% caso queira o hyperref com abntex, comente a linha anterior e descomente a seguinte
%\usepackage[alf,abnt-etal-cite=3,abnt-etal-list=0,abnt-etal-text=emph]{abntex2cite}
%
% caso vc ainda use a versão anterior da abntex, comente a linha incluindo o abntex2cite
% e descomente a próxima linha 
%\usepackage[alf,abnt-etal-cite=3,abnt-etal-list=0,abnt-etal-text=emph]{abntcite}


% redefinindo formatação de títulos de tabelas e figuras


%==============================================================================
% para os fãs do Word, descomente as linhas abaixo
%\sloppy %mais espaço entre as linhas
%\usepackage{identfirst} %identando-se a primeira linha de cada seção
%\noindentfirst % Tire o comentário para manter o padrão do LaTeX.

%==============================================================================
% definido comandos na monografia - não é necessário na sua monografia 
% apenas para exemplificar a definição de novos comandos
\newcommand{\defs}[1]{\textsl{#1}}


% Especificando hifenizações que por ventura LaTeX não saiba fazer
% Por padrão 99,9% dos termos em português devem ser hifenizados corretamente.
\hyphenation{hardware software Li-nux am-bien-te diag-nos-ti-car coor-de-na-ção 
FAE-PE Recovery TelEduc Williams UFLA}

%==============================================================================
% Dados da monografia, capa: autor, titulo, banca, etc... - SUBSTITUA DE ACORDO
%==============================================================================
\author{Joaquim Quinteiro Uchôa}
\title{Uso da Classe Uflamon}
\subtitle{Exemplo para os Usuários}
\engtitle{Use of Uflamon Class}
\engsubtitle{Sample for Users}
\edicao{3$^a$ edição revista, atualizada e ampliada}
\date{2016}
\tipo{Tese apresentada à Universidade Federal de Lavras, como parte das exigências do Programa de Pós-Graduação em Monografia, área de concentração em TCC, para a obtenção do título de Doutor.}
% use \orientador ou \orientadora quando for o caso
\orientador{Prof. DSc. José Orientador}
%\orientadora{}
% use \coorientador ou \coorientadora quando for o caso
\coorientadora{Prof. DSc. Maria Orientadora } % comente se não tiver coorientador
%\coorientador{}
\local{Lavras -- MG}
\bancaum{Prof. MSc. Antônio Banca Um}{UFM}
\bancadois{Prof. DSc. João Banca Dois}{FCO} % comente se sua banca tiver só um professor
\bancatres{Profa. Esp. Eliza Banca Três}{BELMIS}
\bancaquatro{Prof. Esp. Carlos Banca Quatro}{IBGPLUS}
\defesa{30 de Fevereiro de 2016}
%==============================================================================
%##################################################
% Dados para Ficha catalográfica, gerada pelo sistema da Biblioteca da UFLA
% http://www.biblioteca.ufla.br/FichaCatalografica/
% dados para ficha catalográfica
% Elaboração da Ficha Catalográfica
\preparofichacat{Ficha catalográfica elaborada pela Coordenadoria de Processos Técnicos \\ da Biblioteca Universitária da UFLA}
% primeiro autor - como na primeira linha da ficha catalográfica
\fcautor{Uchôa, Joaquim Quinteiro}
% autores, separados por vírgula - na ficha catalográfica, no formato que
% vem após o título e a barra ("/")
\fcautores{Joaquim Quinteiro Uchôa}
% caso trabalho seja ilustrado (figuras, gráficos, tabelas, etc.), 
% então informar por meio do comando a seguir
% caso não seja ilustrado, basta comentá-lo
\fcilustrado{il.}
% dados da edição para a ficha 
\fcedicao{2$^a$ ed. rev., atual. e ampl.}
% tipo do trabalho (tese, dissertação, etc.), de acordo com sistema
% de geração de ficha catalográfica
\fctipo{Tese(doutorado)}
% ano da defesa, só precisa informar se for diferente do ano da publicação
% se forem iguais, comente a linha a seguir
\fcdatadefesa{2016}
% preencher aqui com os dados de catalogação gerados pelo sistema
\fccatalogacao{1. TCC. 2. Monografia. 3. Dissertação. 4. Tese. 5. Trabalho Científico – Normas. I. Universidade Federal de Lavras. II. Título.}
\fcclasi{808.066}

% caso seja preciso diminuir ou aumentar a altura da ficha catalográfica, altere aqui
\alturaFC{8cm}
%##################################################

%\antesfichacat{\noindent Para citar este documento: \\UNIVERSIDADE FEDERAL DE LAVRAS. Biblioteca Universitária. \textbf{Manual de normalização e estrutura de trabalhos acadêmicos: TCC, monografias, dissertações e teses}. 2. ed. rev., atual. e ampl. Lavras, 2015. Disponível em: \url{http://www.biblioteca.ufla.br/wordpress/wpcontent/uploads/bdtd/manual_normalizacao_UFLA.pdf}. Acesso em: data de acesso.}

%\depoisfichacat{\noindent A reprodução e a divulgação total ou parcial deste trabalho são autorizadas, por qualquer meio convencional ou eletrônico, para fins de estudo e pesquisa, desde que citada a fonte.\\
%\newline
%{\small Este documento possui páginas em branco para facilitar a impressão frente-e-verso.}}

%##################################################

%##################################################

% para os exemplos do manual
%\newenvironment{exemplomanual}{
%\vspace{0.5cm}
%\noindent\begin{minipage}{\textwidth}
%\noindent\rule{\textwidth}{0.5pt}
%\vspace{-1cm}
%\begin{flushleft}
%}{
%\end{flushleft}
%\vspace{-0.6cm}
%\noindent\rule{\textwidth}{0.5pt}
%\vspace{0.3cm}
%\end{minipage}
%}

%\newenvironment{exemplomanuallista}{
%\vspace{0.3cm}
%\noindent\begin{minipage}{\textwidth - 0.5cm}
%\noindent\rule{\textwidth}{0.5pt}
%\vspace{-1cm}
%\begin{flushleft}
%}{
%\end{flushleft}
%\vspace{-0.6cm}
%\noindent\rule{\textwidth}{0.5pt}
%\vspace{0.3cm}
%\end{minipage}
%}

% por conta de alguns exemplos
%\usepackage{setspace}

%##################################################

% se vc já defendeu e tem o arquivo escaneado da folha de rosto, 
% descomente e altere o nome do arquivo
%\folhaAprovacaoAssinada{folharosto}

% Aqui começa o documento propriamente dito
\begin{document}

\maketitle

\dedic{Espaço reservado a dedicatória.}     % Dedicatórias\\

\thanks{Espaço reservado aos agradecimentos.}         % Agradecimentos

\epigrafe{ % citação opcional
Espaço reservado a epígrafe.\\
(Autor Desconhecido)}

% palavras-chave
\palchaves{Resumo. Palavras. Representativas.}
\resumo{O resumo deve conter palavras representativas do conteúdo do trabalho, localizadas abaixo do resumo, separadas por dois espaços, antecedidas da expressão palavras-chave. Essas palavras representativas são grafadas com a letra inicial em maiúscula, separadas entre si por ponto.}  % Resumo (digite aqui o resumo)

% keywords devem vir antes do abstract
\keywords{Summary. Words. Representative.} % keywords
\abstract{The abstract should contain representative words of the work content, located below the abstract, separated by two spaces, preceded by the keyword expression. These representative words are spelled with the first letter capitalized, separated by point.}

%##################################################

% Dados do guia
%\begin{titlepage}
%\pagestyle{empty}
%\renewcommand{\baselinestretch}{1}
%\enlargethispage{1.5cm}
%\input{reitoria}
%\cleardoublepage
%\end{titlepage}

%##################################################

% descomente para habilitar a lista desejada
\listoffigures                             % Lista de Figuras
%\listofilustracoes
%\listofgraficos							   % Lista de Gráficos
\listoftables                              % Lista de Tabelas
\listofquadros							   % Lista de Quadros
%\listofexemplos
%\listofteoremas
\tableofcontents                           % Sumário

\clearpage

\pagestyle{ufla}

%==============================================================================
% incluindo os capitulos

\chapter{REFERENCIAL TEÓRICO}
\section{A bananeira}
A bananeira é uma planta de crescimento rápido que necessita de quantidades adequadas de nutrientes disponíveis no solo para seu desenvolvimento e produção. Embora parte das necessidades nutricionais possa ser suprida pelo solo e pelos resíduos das colheitas, frequentemente é necessário aplicar calcário e fertilizantes para garantir uma produção rentável economicamente \cite{culturabanana}.

A quantidade de nutrientes que precisa ser aplicada à variedade de bananeira depende de fatores como o potencial produtivo da planta, a densidade populacional, o estado fitossanitário e, principalmente, o balanço de nutrientes no solo e o sistema radicular, que influenciam a absorção dos nutrientes. A demanda por nutrientes é elevada devido à grande quantidade exportada durante a colheita dos cachos de banana \cite{culturabanana}.

O potássio (K) e o nitrogênio (N) são os nutrientes mais absorvidos e necessários para o crescimento e produção da bananeira, seguidos pelo magnésio (Mg) e pelo cálcio (Ca). Em menor grau de absorção, encontram-se os nutrientes enxofre (S) e fósforo (P) \cite{culturabanana}.

Entre os micronutrientes estudados, o boro (B) e o zinco (Zn) são os mais absorvidos, principalmente pela variedade de bananeira "Terra", seguidos pelo cobre (Cu) \cite{culturabanana}.

A bananeira é uma cultura que demanda uma quantidade significativa de nutrientes durante a adubação, porém também contribui com uma boa quantidade de nutrientes para o solo. É importante ressaltar que a necessidade de nutrientes varia entre as diferentes variedades de bananeira \cite{culturabanana}.

\section{Grande Naine}
A Grande Naine é uma variedade mutante da banana Nanica, originária da ilha de Martinica. Embora seu pseudocaule seja semelhante ao da Nanica, possui manchas escuras e tamanho médio, com altura variando entre 2,5 e 3 metros. A roseta foliar é um pouco mais solta, o que diminui o problema de engasgamento da inflorescência. As folhas são um pouco maiores do que as da Nanica, apresentando cores menos intensas e menor cerosidade. O cacho é mais peludo e um pouco mais longo e tem uma forma ligeiramente cônica, com frutos delgados, longos e curvados, tendo a extremidade arredondada, pedicelos curtos e polpa madura com um sabor muito doce. As primeiras fileiras de bananas no cacho têm almofadas curtas e as bananas são menos curvas em comparação com a banana Nanica, o que facilita seu acondicionamento em caixas de embalagem. A ráquis é reta, sendo comum que o primeiro terço esteja quase livre de resíduos florais masculinos, enquanto o restante está preenchido com esses resíduos parcialmente cobertos por brácteas. O coração da Grande Naine é maior do que o da Nanica. O peso fresco do cacho varia de 15kg a 30kg, dependendo dos níveis de nutrição aplicados, uma vez que essa cultivar responde bem à adubação \cite{cultivodebananeira}.

É uma das variedades cultivadas para a comercialização no mercado internacional. Seu porte menor em comparação com a bananeira Nanicão ajuda a reduzir os danos causados por ventos fortes. No entanto, as perdas devido às últimas fileiras de bananas não estarem dentro dos padrões são significativas. Para compensar essa deficiência, às vezes é necessário remover até as três últimas fileiras. A cultivar possui alta capacidade de resposta em condições de alta tecnologia, mas não tem a mesma resistência que a cultivar Nanicão \cite{cultivodebananeira}.

Essa variedade apresenta suscetibilidade à sigatoka amarela \cite {rocha2008epidemiologia, cultivodebananeira} e também a sigatoka negra e também aos nematoides e ao moleque-da-bananeira, entretanto é resistente ao fungo da fusariose \cite{cultivodebananeira}

\section{PRATA GORUTUBA}
A "Prata Gorutuba" é um tipo de banana selecionado a partir de uma mutação espontânea da variedade Prata Anã, que é cultivada no Norte de Minas Gerais \cite{pratagorutuba1}. De acordo com Lopez \& Espinosa (1995), as bananeiras têm alta eficiência em produzir uma grande quantidade de biomassa em um curto período de tempo, o que demanda altas taxas de nutrientes. Portanto, garantir a nutrição adequada das bananeiras é de extrema importância para a cadeia produtiva.

A grande quantidade de fertilizantes necessária se deve não apenas à alta demanda de nutrientes absorvidos e exportados pelos frutos, mas também ao fato de que os solos da maioria das regiões produtoras geralmente apresentam baixa fertilidade (Borges \& Oliveira, 2000). A bananeira é uma planta sensível a desequilíbrios nutricionais. Portanto, é essencial manter o equilíbrio dos nutrientes no solo, evitando assim o consumo excessivo de um elemento, o que pode levar à deficiência de outros (Gutierrez, 1983). A demanda por nutrientes pela planta depende da taxa de crescimento e da eficiência em converter os nutrientes absorvidos em biomassa.

\section{Maçã}
Essa cultivar possui frutos de excelente qualidade e é altamente aceita no mercado consumidor devido ao seu aroma e sabor semelhantes à maçã. A planta tem um porte que varia de 3,5 a 4 metros de altura e um diâmetro de 30 a 35 centímetros na base do pseudocaule. Na parte mais alta do pseudocaule, há algumas manchas esporádicas e irregulares, que são quase pretas. As folhas são levemente opacas e têm uma disposição inclinada para baixo, dando à planta a aparência de um guarda-chuva aberto. A maior curvatura das folhas ocorre nos primeiros metros a partir do pecíolo. A inflorescência é pequena e o cacho tem tamanho médio e é relativamente fino \cite{cultivodebananeira}.

O peso fresco do cacho varia de 10kg a 12kg e é composto por seis a oito pencas. As pencas estão bem espaçadas ao longo da ráquis, sendo que as primeiras têm em média 18 bananas, enquanto as últimas têm de seis a oito bananas. O comprimento dos frutos varia de 10cm a 18cm. A casca é fina, exalando um suave aroma, e quando maduras, as bananas apresentam uma coloração amarela intensa. A polpa é levemente adocicada, extremamente macia, quase com uma textura farinácea, e tem uma coloração branca. Ocasionalmente, podem ser encontradas sementes férteis no interior dos frutos. A ráquis é bastante longa, de espessura mediana e está livre de resíduos florais \cite{cultivodebananeira}.

Um dos principais desafios para o cultivo dessa cultivar é sua alta suscetibilidade à fusariose-da-bananeira, exigindo uma adubação adequada com Zn, Ca, Mg e P. Essa cultivar tem sido utilizada como cultura de desbravamento em regiões do interior do Brasil devido ao seu período de produção muito curto, limitado a apenas uma a três colheitas, devido à presença dessa doença fúngica. Recomenda-se realizar o plantio em solos que não tenham sido utilizados para o cultivo de bananeiras nos últimos trinta anos, e preferencialmente utilizando mudas produzidas por meio de biotecnologia. Além disso, essa cultivar é suscetível ao ataque do moleque-da-bananeira e nematoides, mas apresenta resistência à sigatoka amarela \cite{cultivodebananeira}.

\section{BRS Princesa}
É um híbrido desenvolvido pela Embrapa - Mandioca e Fruticultura. Trata-se do cruzamento entre as variedades Yangambi n.° 2 (AAB) e M53 (AA). Os frutos desse híbrido se assemelham aos da cultivar Maçã \cite{cultivodebananeira}. 

As plantas têm um porte alto, variando de 3 a 5,5 metros, com um pseudocaule vigoroso que mede entre 20 a 35 centímetros de diâmetro. A massa fresca do cacho varia de 10 a 16 quilogramas, com frutos que têm um comprimento de 10 a 15 centímetros e um diâmetro de 30 a 35 milímetros. O sabor desses frutos é semelhante ao da banana Maçã. Um detalhe importante nesse híbrido é as resistências ao mal-do-panamá e as sigatokas (amarela e negra) \cite{cultivodebananeira}.


\section{Bokashi}
O bokashi é um termo japonês que significa "matéria fermentada" ou "matéria orgânica fermentada". É um método tradicional de compostagem utilizado na agricultura orgânica e na jardinagem para melhorar a fertilidade do solo e fornecer nutrientes para as plantas \cite{bokashi1}.

O processo de produção do bokashi envolve a fermentação de resíduos orgânicos, como restos de alimentos, aparas de grama, folhas, cascas de frutas e vegetais, entre outros materiais ricos em nutrientes. Esses resíduos são misturados com um inoculante, que geralmente contém microorganismos eficientes (como bactérias e fungos benéficos) e materiais que estimulam a decomposição \cite{bokashi1,bokashi3}.

Durante a fermentação, os microorganismos presentes no bokashi quebram a matéria orgânica, convertendo-a em nutrientes solúveis que as plantas podem absorver mais facilmente. Além disso, o bokashi ajuda a melhorar a estrutura do solo, aumentar a capacidade de retenção de água e promover a atividade microbiana benéfica no solo \cite{bokashi1}.

O bokashi pode ser usado como um adubo orgânico, sendo adicionado diretamente ao solo ou utilizado como cobertura do solo. Também pode ser usado na compostagem tradicional, misturado com outros materiais orgânicos para acelerar o processo de decomposição \cite{bokashi1,Hafle2009}.

O uso regular de bokashi pode melhorar a saúde do solo, aumentar a produtividade das plantas, reduzir a necessidade de fertilizantes químicos e contribuir para a sustentabilidade agrícola. É uma técnica popular entre os praticantes da agricultura orgânica e da permacultura \cite{bokashi2}.

\chapter{Materiais e Métodos}

O trabalho foi conduzido de agosto de 2023 à dezembro de 2023, no setor de fruticultura, pertencente ao departamento de agronomia da Universidade Federal de Lavras --- UFLA, que está localizada no municípo de Lavras, que fica a 855 \unit{\metre} de altitude e 21º15'00' de latitude sul. O clima da região é subtropical.

\section{Material Genético}
Foram utilizadas quatro cultivares,adquiridas da Empresa Multiplanta,om cerca de 7 \unit{\centi\metre} de altura. Essas mudas foram transplantadas  para tubetes contendo substrato comercial da marca Carolina Soil, sendo 70 mudas da variedade Grande Naine, 70 da Prata-Gorutuba, 21 da Maçã e 21 da Princesa. Todas as mudas receberam os mesmos tratamentos e condições ambientais durante a condução do experimento dentro de uma casa de vegetação climatizada com uma chuva ao longo do dia.

\section{Organização do experimento}
O experimento foi realizado utilizando o delineamento inteiramente casualisado (DIC), com sete tratamentos cada cultivar. Sendo que a variedade Grande Naine teve 10 plantas para cada tratamento, Prata-Gorutuba também teve 10 plantas para cada tratamento, Maçã e Princesa tiveram três plantas para cada tratamento.

\section{Lista de tratamentos}

\begin{itemize}
	\item T1: 85 \unit{\gram} de substrato e 5 \unit{\gram} do adubo Bokashi;
	\item T2: 85 \unit{\gram} de substrato, 5 \unit{\gram} do adubo Bokashi e 5 \unit{\gram} do condicionador de solo lithothamnium;
	\item T3: 85 \unit{\gram} de substrato, e 5 \unit{\gram} do adubo Bokashi e 5 \unit{\gram} do adubo Baks da Empresa Verde;
	\item T4: 57 \unit{\gram} de substrato e 28 \unit{\gram} do adubo natural de gado;
	\item T5: 57 \unit{\gram} de substrato, 28 \unit{\gram} do adubo natural de gado e 5 \unit{\gram} do condicionador de solo lithothamnium;
	\item T6: 57 \unit{\gram} de substrato, 28 \unit{\gram} do adubo natural de gado e 5 \unit{\gram} do adubo Baks da Empresa Verde;
	\item T7: 85 \unit{\gram} de substrato e 5 \unit{\gram} de adubo químico.
\end{itemize}

\section{Características Avaliadas}
As seguintes características foram avaliadas: Altura de planta, diâmetro do pseudocaule, número de folhas, largura e comprimento de folha.

\section{Estatística}
O software R foi utilizado para realizar a análise estatística dos dados coletados. A avaliação dos dados foi realizada utilizando três técnicas principais: análise de variância, teste de Shapiro-Wilk e teste de Levene. Essas análises desempenharam um papel fundamental na obtenção de resultados confiáveis e na interpretação dos efeitos e diferenças observadas.

A análise de variância (ANOVA) foi aplicada para examinar a existência de diferenças significativas entre as médias de diferentes tratamentos de cada variedade de banana. Esse teste estatístico permitiu-me determinar se as diferenças observadas entre os grupos eram estatisticamente significativas, ajudando a estabelecer relações causais e identificar os fatores que contribuíram para as variações observadas.

Além disso, utilizei o teste de Shapiro-Wilk para avaliar a normalidade da distribuição dos dados em cada grupo. Esse teste é amplamente utilizado para verificar se uma amostra segue uma distribuição normal. Ao realizar o teste de Shapiro-Wilk, pude verificar se os pressupostos estatísticos subjacentes à análise de variância foram atendidos, fornecendo uma validação importante para a aplicação do teste.

Adicionalmente, empreguei o teste de Levene para avaliar a homogeneidade das variâncias entre os grupos. Esse teste permite verificar se as variâncias são iguais entre os grupos e é importante para garantir que a análise de variância seja aplicada corretamente. Ao identificar diferenças significativas nas variâncias entre os grupos, pude tomar medidas adequadas para corrigir possíveis violações dos pressupostos estatísticos e garantir resultados mais precisos.

Para lidar com as características dos dados que não apresentavam distribuição normal, foi aplicada a transformação Box-Cox. Essa transformação é uma técnica estatística que é aplicada às variáveis não normais com o objetivo de ajustar sua distribuição para uma forma mais próxima da normalidade. Essa transformação é baseada em uma função que depende de um parâmetro lambda (\λ\).

%\chapter{INTRODUÇÃO}

A bananeira (Musa spp.) é uma das espécies de plantas frutíferas mais produzidas no mundo, sendo produzida na maioria dos países tropicais. Em 2021, a produção mundial atingiu aproximadamente 125 milhões de toneladas, com a Índia sendo o principal país produtor. Em segundo lugar, com 9,4\% da produção total, vem a china. No contexto da produção global, o Brasil se destaca como o quarto maior produtor, representando 5,5\% da produção total. Essa posição ressalta a significativa contribuição do país para o cenário mundial em termos de produção, evidenciando sua importância econômica e sua participação ativa no mercado internacional \cite{banana}. 

A banana é uma fruta que tem o sabor mediamente doce e textura firme \cite{MATSUURA2004}. Acredita-se que a banana seja nativa do sudeste da ásia \cite{1956TaOo}. É uma das frutas mais consumidas no Brasil, com uma produção nacional próxima de 7 milhões de toneladas no ano de 2021 em uma área de mais de 456 mil hectares \cite{banana} Ela também é a fruta fresca mais consumida no mundo. No Brasil o setor gera mais de 500.000 empregos diretos, em Minas Gerais são 60.000 postos de trabalho.

No entanto, a cultura da banana enfrenta diversos desafios, como pragas, doenças, deficiências nutricionais e baixa produtividade \cite{nogueira2013bananicultura}. Uma forma de melhorar a qualidade e a quantidade da produção é o uso de adubos orgânicos, que fornecem nutrientes essenciais para as plantas, além de melhorar as propriedades físicas, químicas e biológicas do solo.

Um dos adubos orgânicos que vem ganhando destaque é o bokashi, que significa "matéria orgânica fermentada" em japonês. O bokashi é um composto obtido pela mistura de diversos materiais orgânicos, como farelos, esterco, cinzas, terra e micro-organismos eficazes (EM), que são responsáveis pela fermentação anaeróbica do material. O bokashi apresenta diversas vantagens, como a rápida decomposição, a liberação controlada de nutrientes, a inoculação de micro-organismos benéficos no solo e a redução de odores e patógenos.

O objetivo deste trabalho foi avaliar o efeito de diferentes tipos de adubos, incluindo o bokashi, no crescimento e no desenvolvimento de mudas de banana. 
%²: Blog | Broto: Seu Jeito Digital de Fazer Agro! (2021). Produção de banana no Brasil. https://blog.broto.com.br/producao-de-banana-no-brasil/

%⁴: Caracterização dos principais polos de produção de banana no Brasil ... (2015). Embrapa Mandioca e Fruticultura - Documentos. https://www.embrapa.br/busca-de-publicacoes/-/publicacao/967182/caracterizacao-dos-principais-polos-de-producao-de-banana-no-brasil

%Origem: conversação com o Bing, 06/07/2023
%(1) TABELA - Produção brasileira de banana em 2021 Área Colhida Produção .... http://www.cnpmf.embrapa.br/Base_de_Dados/index_pdf/dados/brasil/banana/b1_banana.pdf.
%(2) Caracterização dos principais polos de produção de banana no Brasil .... https://www.embrapa.br/busca-de-publicacoes/-/publicacao/967182/caracterizacao-dos-principais-polos-de-producao-de-banana-no-brasil.
%(3) Blog | Broto: Seu Jeito Digital de Fazer Agro!. https://blog.broto.com.br/producao-de-banana-no-brasil/.
%(4) Banana - Portal Embrapa. https://www.embrapa.br/mandioca-e-fruticultura/cultivos/banana.
%\include{elementos}
%\chapter{CONCLUSÃO}
Não foi encontrado diferenças nas características avaliadas entre os tratamentos nas respectivas mudas de bananeira.

%==============================================================================
% Incluindo bibliografia
%\bibliographystyle{plain}             % estilo para labels em numeros
%\bibliographystyle{alpha}             % estilo para labels em iniciais
\bibliographystyle{abntex2-alf}           % estilo para referências usando ABNT, 
                                       % precisa instalar o abntex para usar!!!

%inclui Referências Bibliográficas

\referencias
\bibliography{database/dados}
%==============================================================================
% Incluindo anexos num1erados com letras maiusculas.
%\apendices
\include{apendice1}


%==============================================================================
% Fim do texto
\end{document}
